\documentclass[11pt]{article}
\usepackage{amsfonts}
\usepackage{xspace}
\usepackage{url}
\usepackage{graphicx}
\usepackage[headheight=90pt,headsep=12pt,includeheadfoot,top=24pt,left=1in,footskip=24pt]{geometry}
\usepackage{verbatim}
\usepackage{fancyvrb}
\usepackage{amsmath, amsthm, amssymb}
\usepackage{fancyhdr}
\usepackage{xcolor}
\usepackage[xetex,bookmarks=true,pagebackref=true,linktocpage=true]{hyperref}
%\usepackage[style=list]{glossary}

\newcommand{\titleline}{Generating Formal Models with the Java Symbolic
Simulator}

\hypersetup{%
   pdftitle     = \titleline,
   pdfkeywords  = {Java, JVM, Symbolic Simulation, Cryptography, Programming, Formal Methods, Formal Models},
   pdfauthor    = {Galois, Inc.},
   pdfpagemode  = UseOutlines
}

%\input{utils/Indexes.tex}
%\input{utils/GlossaryItems.tex}
%\input{utils/trickery.tex}

% fonts
\usepackage{fontspec}
\usepackage{xunicode}
\usepackage{xltxtra}
%\setmainfont[Mapping=tex-text]{Garamond}
\setsansfont[Mapping=tex-text]{Optima}
\usepackage{sectsty}
\allsectionsfont{\sffamily}

\newcommand{\todo}[1]{\begin{center}\framebox{\begin{minipage}{0.8\textwidth}{{\bf TODO:} #1}\end{minipage}}\end{center}}

\DefineVerbatimEnvironment{code}{Verbatim}{}

% various little text sections:
\newcommand{\draftdate}{DRAFT of \today}
% \setlength{\voffset}{-1cm}
% \setlength{\headsep}{2cm}
% \setlength{\headheight}{15.2pt}
% \renewcommand{\headrulewidth}{0pt} % no line on top
% \renewcommand{\footrulewidth}{.5pt} % line on bottom
% %\renewcommand{\chaptermark}[1]{\markboth{#1}{}}
% %\renewcommand{\sectionmark}[1]{\markright{#1}{}}
% \cfoot{}
% \fancyfoot[LE,RO]{\fancyplain{}{\textsf{\thepage}}}
% \fancyfoot[LO,RE]{\fancyplain{}{\textsf{\copyright\ 2010, Galois, Inc.}}}
% %\fancyhead[LE]{\fancyplain{}{\textsf{\draftdate}}}
% %\fancyhead[RO]{\fancyplain{}{\textsf{DO NOT DISTRIBUTE!}}}
% \fancyhead[LE]{\fancyplain{}{\textsf{\titleline}}}
% \fancyhead[RO]{\fancyplain{}{\textsf{\titleline}}}
% \fancyhead[LO,RE]{}
% \pagestyle{fancyplain}

%\makeglossary
%\makeindex

\title{\LARGE{\bf \titleline}}
\author{\\$ $\\$ $\\
        Galois, Inc.\\
        421 SW 6th Ave., Suite 300\\Portland, OR 97204}
\date{
\vspace*{2cm}$ $\\
\includegraphics[height=1in]{images/Galois_logo_blue_box.pdf}
\vspace*{2cm}$ $\\
%\draftdate
%\\
%PLEASE DO NOT DISTRIBUTE!
}

\pagestyle{fancy}
\fancyheadoffset[l]{1in}
\fancyheadoffset[r]{0.2in}
\fancyhead{}%clear
\fancyfoot{}%clear
\lhead{\sffamily\hspace{24pt}\includegraphics[height=1in]{images/Galois_logo_blue_box.pdf}}
\rhead{\sffamily\titleline\\\vspace{12pt}}
\rfoot{\sffamily\thepage}
\lfoot{\sffamily\copyright 2010 Galois, Inc.}
\renewcommand{\headrulewidth}{0pt}
\renewcommand{\footrulewidth}{1pt}

\begin{document}

\maketitle
\thispagestyle{empty}
\newpage

\section{Introduction}

This document describes how to use the Galois Java Symbolic Simulator's
command-line interface, \texttt{jss}, to generate a formal model of a
cryptographic algorithm written in Java, and to compare that model
against a reference specification to verify that the Java implementation
is correct. This brings benefits such as allowing programmers to
experiment with efficient, customized implementations of an algorithm
while retaining confidence that the changes do not affect the overall
functionality.

We assume knowledge of Java, and a general understanding of
cryptography. However, we do not assume familiarity with symbolic
simulation, formal modeling, or theorem proving.  Additionally, we
assume that the user has installed the Cryptol tool set and the ABC
logic synthesis system from UC Berkeley if she wishes to complete the
equivalence checking.  Installation and configuration of those tools is
outside the scope of this tutorial.  However, instructions on setting up
the environment for \texttt{jss} (i.e., installing the JDK 1.6 and
Bouncy Castle class files ) can be found in the next section.

In the examples of interaction with the simulator and other tools, lines
beginning with a hash mark (\texttt{\#}) or short text followed by an
angle bracket (such as \texttt{abc 01>}) indicate command-line prompts,
and the following text is input provided by the user. All other
monospaced text is the output of the associated tool. 

\section{Setting up the Environment}
\label{sec:setup}
 
The \texttt{jss} tool requires the classes that form Java's runtime in
order to execute most useful Java code; the current \texttt{jss} release
has been tested with JDK 1.6.  Furthermore, this tutorial relies on the
Bouncy Castle implementation of MD5, and so some supporting class files
are required for that as well.  Fortunately, this is as simple as
installing or locating two Java JAR files to provide as arguments to
\texttt{jss}.

\subsection{The JDK JAR File}

The location of the JDK JAR file is platform-dependent.

\vspace{5mm}

\emph{If you're on Windows or Linux}, download the appropriate Java SDK
6u22 via the web forms at:

\begin{center}
\texttt{http://www.oracle.com/technetwork/java/javase/downloads/index.html}.
\end{center}

The JDK jar (called \texttt{rt.jar}) can be found at:

\begin{center}
\texttt{<JDK INSTALLATION ROOT>\textbackslash jre\textbackslash lib\textbackslash rt.jar}
\end{center}

e.g.: \texttt{C:\textbackslash Program Files\textbackslash Java\textbackslash jdk1.6.0\_22\textbackslash jre\textbackslash lib\textbackslash rt.jar}

\vspace{5mm}

On Linux, the path is as above, rooted at the location where the Java
SDK 6u22 was extracted.

\vspace{5mm}

\emph{If you're on Mac OS X}, the JDK should already be installed along
with the platform developer tools.  Executing the shell command
\texttt{locate classes.jar} should reveal the location of a jdk1.6
runtime jar suitable for use with \texttt{jss}.  E.g.,

\begin{center}
\texttt{/System/Library/Java/JavaVirtualMachines/1.6.0.jdk/Contents/Classes/classes.jar}
\end{center}

\vspace{5mm}

In either case, the identified file (either \texttt{rt.jar} or
\texttt{classes.jar}) should be the file substituted for
\texttt{JDK\_JAR} in the command line invocations shown in subsequent
sections.

\subsection{The Bouncy Castle JAR File}

To obtain code needed for the Bouncy Castle MD5 implementation employed
in this tutorial, download the Bouncy Castle JAR file via

\begin{center}
\texttt{http://www.bouncycastle.org/download/bcprov-jdk16-145.jar}
\end{center}

This should be the file substituted for \texttt{<BC\_JAR>} in the command
line invocations shown in subsequent sections.

\section{Symbolic Simulation}

The Java Symbolic Simulator takes the place of a standard Java virtual
machine but makes it possible to reason about the behavior of programs
on a wide range of potential inputs, rather than a fixed set of test
vectors. The standard Java virtual machine executes programs on
concrete input values, producing concrete results. Symbolic simulation
works somewhat similarly, but allows inputs to take the form of
symbolic variables that represent arbitrary, unknown values. The
result is then a mathematical formula that describes the output of the
program in terms of the symbolic input variables.

Given a formula representing a program's output, we can then either
evaluate that formula with specific values in place of the symbolic
input variables to get concrete output values, or compare the formula
to another, using known mathematical transformations to prove that
the two are equivalent.

One downside of symbolic simulation, however, is that it cannot easily
handle interaction with the outside world. If a program simply takes an
input, performs some computation, and produces some output, symbolic
simulation can reliably construct a model of that computation.
Cryptographic algorithms typically fall into this category. In cases
where a program does some computation, produces some output, reads some
more input, and continues the computation based on the new information,
we either need a model of the outside world, or to assume that the input
could be completely arbitrary, and reason about what the program would
do for any possible input.

The Java Symbolic Simulator can evaluate simple output methods, such
as the ubiquitous \texttt{System.out.println}, but only when the value
passed as an argument is completely known. If its value depends on the
value of a symbolic input variable, it cannot be printed.

Instead, the Java Symbolic Simulator provides a set of special methods
for performing operations on symbolic values, and for emitting the
formal model representing symbolic values of interest. The rest of this
tutorial will demonstrate how to use these methods to generate a formal
model of the MD5 digest algorithm, and then compare this model to a
reference specification.

\section{Supplying Symbolic Input to MD5}

In the directory containing this tutorial, there is an example source
file, \texttt{JavaMD5.java}, which creates a simple wrapper around
the MD5 digest function from the The Legion of the Bouncy Castle.
We will use this source file as a running example, and step through
what each line means in the context of symbolic simulation.

The \texttt{JavaMD5} class first imports from
\texttt{org.bouncycastle.crypto.digests} to get access to the MD5
digest code, and from \texttt{com.galois.symbolic} for access to the
Java Symbolic Simulator's special methods for using symbolic values.

The first variable declaration in the \texttt{main} function is the
one with the most relevance to symbolic simulation.

\begin{code}
byte[] msg = Symbolic.freshByteArray(16);
\end{code}

This declaration creates a new array, but one with completely symbolic
contents. Its size is fixed (16 elements), but each of those elements
is a symbolic term representing an arbitrary byte value.

The next two declarations are completely standard Java, and create a
new array of 16 bytes along with a \texttt{MD5Digest} object.

\begin{code}
byte[] out = new byte[16];
MD5Digest digest = new MD5Digest();
\end{code}

These variables have no direct connection to symbolic values upon
declaration, but the values stored in them will be symbolic if they
depend on the values in the \texttt{msg} array.

Next, calculation of the message digest happens in the usual way, by
calling two methods in the Bouncy Castle API, just as they would occur
in a typical application.

\begin{code}
digest.update(msg, 0, msg.length);
digest.doFinal(out, 0);
\end{code}

The next and final statement does the work of creating a formal model
from the MD5 digest code. This method instructs the symbolic simulator
to generate a formula that describes how the value of the variable
\texttt{out} depends on the values of the elements of \texttt{msg} and
then write that formula to a file called \texttt{JavaMD5.aig}.

\begin{code}
Symbolic.writeAiger("JavaMD5.aig", out);
\end{code}

Ultimately, we want to find a formal model that describes the output
of the digest function, in terms of whatever symbolic inputs it
happens to depend on. In the case of MD5, this includes every byte of
the input message. However, for some algorithms, it could include only
a subset of the symbolic variables in the program.

The formal model that the simulator generates takes the form of an
And-Inverter Graph (AIG), which is a way of representing a boolean
function purely in terms of the logical operations ``and'' and
``not''. The simplicity of this representation makes the models easy
to reason about, and to compare to models from other sources.
However, the same simplicity means that the model files can be very
large in comparison to the input source code.

\section{Running the Simulator}

To generate a formal model from the example described in the previous
section, we can use the \texttt{jss} command, which forms the
command-line front end of the Java Symbolic Simulator. It needs to know
where to find the Java class files for the standard library, the Bouncy
Castle encryption libraries, and the Galois Symbolic libraries.  The
latter should be found automatically by the tool as long as the
\texttt{jss} is being invoked from the distribution directory hierarchy.  

The following command will run \texttt{jss} to create a formal model;
substitute \texttt{<JDK\_JAR>} and \texttt{<BC\_JAR>} with the appropriate
JAR file names for your system, as described in section \ref{sec:setup}.

\begin{code}
# jss -c . -j <BC_JAR>:<JDK_JAR> JavaMD5
\end{code}

{\bf NB:} The colon character (\texttt{:}) is used to delimit JAR file
names on UNIX systems (including Mac OS X).  On Windows systems, the
semicolon character (\texttt{;}) is used instead.  For example, assuming
that \texttt{jss} is in the \texttt{\%PATH\%}, an invocation on a
Windows system may look like:

\begin{scriptsize}
\begin{code}
jss -c . -j C:\myjars\bcprov-jdk16-145.jar;"C:\Program Files\Java\jdk1.6.0_22\jre\lib\rt.jar" JavaMD5
\end{code}
\end{scriptsize}


This will result in a file called \texttt{JavaMD5.aig} that can be
further analyzed using a variety of tools, including the Galois
Cryptol tool set, and the ABC logic synthesis system from UC
Berkeley.

\section{Verifying the Formal Model Using Cryptol}

One easy way to verify a Java implementation against a reference
specification is through the Cryptol tool set. Cryptol is a
domain-specific language created by Galois for the purpose of writing
high-level but precise specifications of cryptographic
algorithms~\cite{cryptol}. The Cryptol tool set has built-in support
for checking the equivalence of different Cryptol implementations, as
well as comparing Cryptol implementations to external formal models.

This tutorial comes with two Cryptol files, \texttt{MD5.cry} and
\texttt{compare-md5.cry}. The former is a Cryptol specification of the
MD5 algorithm. In particular, it contains the function
\texttt{md5\_ref}, which is specialized to operate on 16-byte
messages, and should have equivalent functionality to the Bouncy
Castle implementation.

To compare the functionality of the two implementations, we have
several options. As mentioned earlier, formal models can be evaluated
on concrete inputs, or compared to other formal models using proof
techniques to show equivalence for all possible inputs. The contents
of \texttt{compare-md5.cry} show how to compare the formal model of
the Java implementation against the Cryptol reference specification.

\begin{code}
include "MD5.cry";
extern AIG md5_java("JavaMD5.aig") : [16][8] -> [128];
theorem MatchesRef : {m}. md5_java m == md5_ref m;
\end{code}

The first line imports the contents of \texttt{MD5.cry}, for access to
the \texttt{md5\_ref} function. Then, the \texttt{extern AIG} line
makes the contents of \texttt{JavaMD5.aig} available as a function
called \texttt{md5\_java} that takes 16 8-bit values as input and
produces one 128-bit value as output. Finally, the third line states a
theorem: that the functions \texttt{md5\_java} and \texttt{md5\_ref}
should produce the same output for all possible inputs.

We can load \texttt{compare-md5.cry} into the Cryptol tool set,
yielding the following output:

\begin{code}
# cryptol compare-md5.cry
Cryptol version 1.9.0, Copyright (C) 2004-2010 Galois, Inc.
                                           www.cryptol.net 
Type :? for help
Loading "compare-md5.cry"..
  Including "MD5.cry".. Checking types..
  Loading extern aig from "JavaMD5.aig".. Processing.. Done!
*** Auto quickchecking 1 theorem.
*** Checking "MatchesRef" ["compare-md5.cry", line 4, col 1]
Checking case 100 of 100 (100.00%) 
100 tests passed OK
[Coverage: 0.00%. (100/340282366920938463463374607431768211456)]
compare-md5> 
\end{code}

By default, the Cryptol interpreter processes every \texttt{theorem}
declaration by automatically evaluating the associated expression on a
series of random values, and ensuring that it always yields ``true''.
In this case, it tried 100 random values, and the two functions
yielded the same output in each case. However, the number of possible
16-byte inputs is immense, so 100 test cases barely scratches the
surface.

To gain a higher degree of confidence that the functions do have the
same functionality for all possible inputs, we can attempt to prove
their equivalence deductively. From Cryptol's command line:

\begin{code}
compare-md5> :set symbolic
compare-md5> :prove MatchesRef
Q.E.D.
compare-md5> :fm md5_ref "MD5-ref.aig"
\end{code}

This tells the Cryptol interpreter to switch to symbolic simulation
mode (which is one way it can generate formal models from Cryptol
functions) and then attempt to prove the theorem named
\texttt{MatchesRef}. On a reasonably modern machine (as of October
2010), the proof should complete in several minutes. The output
\texttt{Q.E.D.} means that the proof was successful.

Finally, the \texttt{:fm} command tells the interpreter to generate a
formal model of the \texttt{md5\_ref} function and store it in the
file \texttt{MD5-ref.aig}. We can then use this formal model to
perform the same proof using an external tool such as ABC, as
described next.

\section{Verifying the Formal Model Using ABC}

ABC is a tool for logic synthesis and verification developed by
researchers at UC Berkeley~\cite{abc}. It can perform a wide variety
of transformations and queries on logic circuits, including those in
the AIG form discussed earlier.

As an alternative approach to the equivalence check from the previous
section, we can use the \texttt{cec} command in ABC to attempt to
prove the model generated by the symbolic simulator equivalent to the
model generated from the Cryptol specification.

\begin{code}
# abc
UC Berkeley, ABC 1.01 (compiled Oct 26 2010 13:07:15)
abc 01> cec ./JavaMD5.aig ./MD5-ref.aig
Networks are equivalent.
abc 01> 
\end{code}

\section{Evaluating Formal Models on Concrete Inputs}

So far, we have demonstrated how to use the API of the symbolic
simulator to generate formal models, in the form of And-Inverter
Graphs, that describe the symbolic values of particular program
variables, and then use external tools to analyze those formal models.
The API also provides the ability to evaluate a formal model on
specific concrete inputs from within the simulator.

In the example from \texttt{JavaMD5.java}, the variable \texttt{out}
depends on symbolic inputs and is therefore represented by a symbolic
model. However, given concrete values for symbolic inputs that the
output depends on, the model can be reduced to a concrete final value.
Evaluation of a symbolic model to a concrete value uses one of the
\texttt{Symbolic.evalAig} methods, depending on the type of the output
variable of interest.

Because the symbolic model describing an output variable may have
inputs of various different types, the symbolic simulator API provides
a class hierarchy to represent possible input values. The
\texttt{CValue} class represents concrete input values, and has inner
subclasses \texttt{CBool}, \texttt{CByte}, \texttt{CInt}, and
\texttt{CLong}, to represent inputs of a variety of sizes.

Given a model output variable and an array of concrete inputs, the
\texttt{evalAig} function produces a new value of the same type as the
given output variable, but one that is guaranteed to contain a
concrete value from the perspective of the symbolic simulator. The
caller of \texttt{evalAig} is responsible for providing the correct
number of inputs. Otherwise, the simulator may throw a runtime error.

For example, if we replace the call to \texttt{Symbolic.writeAiger}
in the example above with the following code, the simulator will print
out the result of evaluating the symbolic model on a concrete message.

\begin{code}
byte[] result = Symbolic.evalAig(out,
  new CValue[] {
    new CValue.CByte((byte) 0x68), // h
    new CValue.CByte((byte) 0x65), // e
    new CValue.CByte((byte) 0x6c), // l
    new CValue.CByte((byte) 0x6c), // l
    new CValue.CByte((byte) 0x6f), // o
    new CValue.CByte((byte) 0x20), // 
    new CValue.CByte((byte) 0x77), // w
    new CValue.CByte((byte) 0x6f), // o
    new CValue.CByte((byte) 0x72), // r
    new CValue.CByte((byte) 0x6c), // l
    new CValue.CByte((byte) 0x64), // d
    new CValue.CByte((byte) 0x21), // !
    new CValue.CByte((byte) 0x21), // !
    new CValue.CByte((byte) 0x21), // !
    new CValue.CByte((byte) 0x21), // !
    new CValue.CByte((byte) 0x21)  // !
  });
for(int i = 0; i < result.length; i++) {
  System.out.println(result[i]);
}
\end{code}

\bibliographystyle{plain}
\bibliography{bib/jss-tutorial}

\end{document}
