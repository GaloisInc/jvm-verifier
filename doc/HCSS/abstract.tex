\documentclass[12pt]{article}

\title{A Toolsuite for\\Compositional Cryptographic Verification}
\author{Joe Hendrix\\\texttt{jhendrix@galois.com}}

\pagestyle{empty}

\begin{document}

\begin{center}
{\LARGE A Toolsuite for Compositional\\ Cryptographic Equivalence Checking}

\vspace{1.5em}

{\large The Cryptol Team}\\
{\large Galois, Inc\footnote{Corresponding author: Joe Hendrix, \texttt{jhendrix@galois.com}}}

\vspace{1.5em}

\end{center}

\noindent
\textbf{Abstract.}
Cryptographic algorithms represent an essential component of many critical
systems, and there is considerable interest in ensuring the correctness of a
cryptographic implementation on all possible key and messages.  For the
verification of symmetric key ciphers and cryptographic hash functions, there
are verification tools capable of showing this correctness automatically using
symbolic simulation and SAT-based equivalence checking.

Despite these advances, verification of public key cryptography algorithms
remains intractable to fully automatic techniques.  Public key algorithms are
not only much more computationally expensive than private key algorithms, but
they also fundamentally rely on algorithms that are notorious for being a
difficult to automatically verify.  This includes algorithms for
large-word multiplication and field division.

In a recent project, Galois has begun the development of a tool suite, called
the Software Analysis Workbench (SAW), that supports the \emph{compositional}
verification of cryptographic algorithms.  The tool suite uses a common
verification infrastructure that couples a bit-precise type system from the
Cryptol language to rewriting and SAT-based equivalence checking verification
algorithms.  Simpler verification tasks may be handled completely automatically,
while more complex verification examples are enabled by
using the verification results of low-level algorithms as lemmas to simplify
the verification of higher-level algorithms.

%This verification infrastructure has been coupled to a symbolic
%simulator for the JVM, and three verification frontends:
%
%\begin{itemize}
%
% \item A command-line tool \texttt{jss.exe} that provides Java developers
%   with a familiar interface to running the symbolic simulator from within Java.
%
% \item A command-line tool \texttt{sawScript.exe} that supports
%   compositional verification of related Java methods using a custom language
%   for specifying pre and post-conditions.
%
% \item A prototype Haskell DSL that provides interactive programatic access to
%   the symbolic simulator using the Glasgow Haskell Compiler's interactive
%   shell.
%
%\end{itemize}

Although the tool is still in early development, we have had some positive
results with applying the verification to a Java-based implementation of
Elliptic Curve Cryptography (ECC).  In the HCSS talk, we will describe the
development of the SAW tool suite, our verification results, and sketch
directions for future research.  Tools such as SAW will become increasingly
critical as complex public key algorithms such as ECC gain more widespread
adoption.

\end{document}
